\section{Documentation}
This is a template for documentation, containing some simple examples that could be usefull for technical documentation. This entire document can be seen as an example, including the structure of the text.

\subsection{Table}\label{sec:Table}

In this subsection we have an example of a \emph{simple} table. This should be enough for most purposes, but if colors and such are needed that can be arranged too. 

\begin{table}[!htb]\centering
	\begin{tabular}{llr}
		\toprule
		\multicolumn{2}{c}{Name} \\
		\cmidrule(r){1-2}
		First name & Last name & office \\
		\midrule
		John & Doe & $ UK$ \\
		Jesper & Andersen & $DK$ \\
		\bottomrule
	\end{tabular}
	\caption{Example Table}
	\label{tab:exampleTable}
\end{table}

Here is more text referencing \cref{tab:exampleTable} above. \Cref{tab:exampleTable} should start with a capital `T', so remember to use 
\verb$\Cref{}$ instead of \verb$\cref{}$.

\subsection{Include codesnippets}
IHere's an example of how to include part of an .xml file. For this to work, you must either place a start and end marker in your source code file(as in the first example), state the actual linenumbers or use some existing text in the file as markers (second example).
\lstset{language=XML,
	   includerangemarker=false,
	   rangeprefix=<!--\ ,
	   rangesuffix=\ -->}
	   
\lstinputlisting[linerange={Begin\ Identifier0-End\ Identifier0}]{test.mdml.xml}
		        
This was the first example. The syntax highlighting for XML is defined in the documentclass template file. If a keyword isn't highlighted properly, it is because it is not in the list of keywords in the language definition. You can add the keyword to the list in the documentclass template file.

\lstset{language=XML,
	   includerangemarker=false,
	   rangebeginprefix=<,
	   rangeendprefix=</,
	   rangesuffix=>}
	   
\lstinputlisting[linerange={Compact-Compact},
			basicstyle=\small\ttfamily\color{xmlTag}]{test.mdml.xml}

In the second example we changed the font attributes for this listing only. The default font is desribed in the syntax highlight definition for XML in the document template file. 

Further documentation of the listings package is found in the \href{http://mirror.jmu.edu/pub/CTAN/macros/latex/contrib/listings/listings.pdf}{Listings Package Documentation}

\subsection{Other types of references}
A common bibtex file has been defined, called \href{run:/deltek.bib}{deltek.bib}. To reference something, e.g. \cite{cleancode}, put it in this file and share. Remember to run bibTex when you have added entries or citations.

