\section{Documentation}
This is a template for documentation, containing some simple examples that could be usefull for technical documentation. This entire document can be seen as an example, including the structure of the text.

\subsection{Table}

In this subsection we have an example of a \emph{simple} table. This should be enough for most purposes, but if colors and such are needed that can be arranged too. 

\begin{table}[!htb]\centering
	\begin{tabular}{llr}
		\toprule
		\multicolumn{2}{c}{Name} \\
		\cmidrule(r){1-2}
		First name & Last name & office \\
		\midrule
		John & Doe & $ UK$ \\
		Jesper & Andersen & $DK$ \\
		\bottomrule
	\end{tabular}
	\caption{Example Table}
	\label{tab:exampleTable}
\end{table}

Here is more text referencing \cref{tab:exampleTable} above. \Cref{tab:exampleTable} should start with a capital `T', so remember to use 
\verb$\Cref{}$ instead of \verb$\cref{}$.

\subsection{Iclude codesnippets}
IHere's an example of how to include part of an .xml file. For this to work, you must either place a start and end marker in your source code file(as in the first example), state the actual linenumbers or use some existing text in the file as markers (second example).
\lstset{language=XML,
	   includerangemarker=false,
	   rangeprefix=<!--\ ,
	   rangesuffix=\ -->}
	   
\lstinputlisting[linerange={Begin\ Identifier0-End\ Identifier0},
		        breaklines=true,
		        showstringspaces=false]{test.mdml.xml}
		        
This was the first example. The syntax highlighting for xml is defined in the documentclass template file.

\lstset{language=XML,
	   includerangemarker=false,
	   rangeprefix=<,
	   rangesuffix=\>}
	   
\lstinputlisting[linerange={Compact>-Compact>},
		        breaklines=true,
		        showstringspaces=false]{test.mdml.xml}

%\lstinputlisting[linerange={Begin\ Identifier0-End\ Identifier0},
%		        breaklines=true,
%		        showstringspaces=false]{test.mdml.xml}
%\begin{lstlisting}[frame=single]
%\lstinputlisting[linerange={<!--\ Begin\ Identifier0\ -->- 8}]{test.mdml.xml}
%\lstinputlisting[linerange=<!--\ Begin\ Identifier0\ -->-<!--\ End\ Identifier0\ -->, 
%			breaklines=true,
%			showstringspaces=false]{test.mdml.xml}
%\end{lstlisting}
Because it really should!